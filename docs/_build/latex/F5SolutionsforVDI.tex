%% Generated by Sphinx.
\def\sphinxdocclass{report}
\documentclass[letterpaper,10pt,english]{sphinxmanual}
\ifdefined\pdfpxdimen
   \let\sphinxpxdimen\pdfpxdimen\else\newdimen\sphinxpxdimen
\fi \sphinxpxdimen=.75bp\relax

\usepackage[utf8]{inputenc}
\ifdefined\DeclareUnicodeCharacter
 \ifdefined\DeclareUnicodeCharacterAsOptional\else
  \DeclareUnicodeCharacter{00A0}{\nobreakspace}
\fi\fi
\usepackage{cmap}
\usepackage[T1]{fontenc}
\usepackage{amsmath,amssymb,amstext}
\usepackage{babel}
\usepackage{times}
\usepackage[Bjornstrup]{fncychap}
\usepackage[dontkeepoldnames]{sphinx}

\usepackage{geometry}

% Include hyperref last.
\usepackage{hyperref}
% Fix anchor placement for figures with captions.
\usepackage{hypcap}% it must be loaded after hyperref.
% Set up styles of URL: it should be placed after hyperref.
\urlstyle{same}
\addto\captionsenglish{\renewcommand{\contentsname}{Contents:}}

\addto\captionsenglish{\renewcommand{\figurename}{Fig.}}
\addto\captionsenglish{\renewcommand{\tablename}{Table}}
\addto\captionsenglish{\renewcommand{\literalblockname}{Listing}}

\addto\extrasenglish{\def\pageautorefname{page}}

\setcounter{tocdepth}{1}

%% LaTeX preamble.

\usepackage{type1cm}
\usepackage{helvet}
\usepackage{wallpaper}

% Bypass unicode character not supported errors
\usepackage[utf8]{inputenc}

\makeatletter
\def\UTFviii@defined#1{%
  \ifx#1\relax
      ?%
  \else\expandafter
    #1%
  \fi
}
\makeatother

\pagestyle{plain}
\pagenumbering{arabic}

\renewcommand{\familydefault}{\sfdefault}

\definecolor{f5red}{RGB}{235, 28, 35}

\def\frontcoverpage{
  \begin{titlepage}
  \ThisURCornerWallPaper{1.0}{front_cover}
  \vspace*{2.5cm}
  \hspace{4.5cm}
  {\color{f5red} \text{\Large Agility 2017 Hands-on Lab Guide}\par}
  \vspace{.5cm}
  \hspace{4.5cm}
  {\color{white} \text{\huge F5 Solutions for VDI}\par}
  \vspace{0.5cm}
  \hspace{4.5cm}
  {\color{white} \text{\large F5 Networks, Inc.}\par}
  \vfill
  \end{titlepage}
  \newpage
}

\def\backcoverpage{
  \newpage
  \thispagestyle{empty}
  \phantom{100}
  \ThisURCornerWallPaper{1.0}{back_cover}
}

\def\contentspage{
    \tableofcontents
}

%% Disable standard title (but keep PDF info).
\renewcommand{\maketitle}{
  \begingroup
  % These \defs are required to deal with multi-line authors; it
  % changes \\ to ', ' (comma-space), making it pass muster for
  % generating document info in the PDF file.
  \def\\{, }
  \def\and{and }
  \pdfinfo{
    /Title (F5 Solutions for VDI)
    /Author (F5 Networks, Inc.)
  }
  \endgroup
}


\title{F5 Solutions for VDI Documentation}
\date{Jul 11, 2017}
\release{}
\author{F5 Networks, Inc.}
\newcommand{\sphinxlogo}{\vbox{}}
\renewcommand{\releasename}{Release}
\makeindex

\begin{document}

\maketitle

\frontcoverpage
\contentspage

\phantomsection\label{\detokenize{index::doc}}



\chapter{Getting Started}
\label{\detokenize{intro:getting-started}}\label{\detokenize{intro::doc}}
Please follow the instructions provided by the instructor to start your
lab and access your jump host.

\begin{sphinxadmonition}{note}{Note:}
All work for this lab will be performed exclusively from the Windows
jumphost. No installation or interaction with your local system is
required.
\end{sphinxadmonition}


\section{Lab Topology}
\label{\detokenize{intro:lab-topology}}
In the interest of focusing as much time as possible on this solution,
we have provided some resources and basic setup ahead of time. These
are:
\begin{itemize}
\item {} 
The system has been licensed and provisioned for LTM and APM

\item {} 
A Microsoft Active Directory environment has been configured for
authentication

\item {} 
A working VMware Horizon View environment has already been configured

\item {} 
A working Citrix XenDesktop environment has already been configured

\item {} 
Windows workstations with Citrix and View clients will be accessed
using RDP to demonstrate functionality.

\end{itemize}

If you wish to replicate these labs in your lab you will need to build
out the required infrastructure.

\sphinxincludegraphics[width=7.38542in,height=4.13542in]{{image2}.png}


\subsection{Lab Components}
\label{\detokenize{intro:lab-components}}
\begin{sphinxadmonition}{note}{Todo:}
Complete lab components table
\end{sphinxadmonition}

The following table lists VLANS, IP Addresses and Credentials for all
components:


\begin{savenotes}\sphinxattablestart
\centering
\begin{tabular}[t]{|\X{20}{100}|\X{40}{100}|\X{40}{100}|}
\hline
\sphinxstylethead{\sphinxstyletheadfamily 
\sphinxstylestrong{Component}
\unskip}\relax &\sphinxstylethead{\sphinxstyletheadfamily 
\sphinxstylestrong{VLAN/IP Address(es)}
\unskip}\relax &\sphinxstylethead{\sphinxstyletheadfamily 
\sphinxstylestrong{Credentials}
\unskip}\relax \\
\hline\sphinxstylethead{\sphinxstyletheadfamily 
Sample Host
\unskip}\relax &\begin{itemize}
\item {} 
\sphinxstylestrong{Management:} 10.1.1.250

\item {} 
\sphinxstylestrong{Internal:} 10.1.10.250

\item {} 
\sphinxstylestrong{External:} 10.1.20.250

\end{itemize}
&
\sphinxcode{admin}/\sphinxcode{admin}
\\
\hline
\end{tabular}
\par
\sphinxattableend\end{savenotes}


\chapter{201 - VDI the F5 Way}
\label{\detokenize{class2/class1:vdi-the-f5-way}}\label{\detokenize{class2/class1::doc}}
This guide is intended to compliment lecture material provided during the 201
course and iApp configuration guide that can be referred to after Agility.


\section{Module 1 - Solutions for VMware View}
\label{\detokenize{class2/module1/module1:module-1-solutions-for-vmware-view}}\label{\detokenize{class2/module1/module1::doc}}
The purpose of this module is to build out 3 basic VMware View
architectures leveraging F5 load balancing and authentication
functionality. Each student will access a separate instance of a common
blueprint in the Ravello cloud.

Note: Each student will be given IP addresses for the “Internal”
(corporate-pc) and “External”(home-pc) Windows workstation clients. The
student will use RDP functionality on their personal laptop to access
these devices to test the environment. All other addresses will be
common across all labs

Objective:
\begin{itemize}
\item {} 
Construct VMware View implementations with F5 LTM and APM software
modules

\item {} 
Familiarize student with F5 iApp templates

\end{itemize}

Lab Requirements:
\begin{itemize}
\item {} 
Laptop with RDP functionality

\end{itemize}

Estimated completion time: 60 Minutes


\subsection{Lab 1 \textendash{} Solutions for VMware View}
\label{\detokenize{class2/module1/lab1:lab-1-solutions-for-vmware-view}}\label{\detokenize{class2/module1/lab1::doc}}

\subsubsection{TASK 1 \textendash{} Access VMware View Desktop environment without F5}
\label{\detokenize{class2/module1/lab1:task-1-access-vmware-view-desktop-environment-without-f5}}
Test the functional VMware view environment using the internal
Connection Servers (Internal use case without F5 integration)

\sphinxincludegraphics[width=5.40625in,height=3.04167in]{{image3}.png}

Figure 1 - Accessing internal View Desktop\textendash{}

Access the VDI with a client on the internal network. The workstation
will be preconfigured to initiate the connection through a specific
connection server. Security servers are not used by internal VDI users
\begin{enumerate}
\item {} 
Use the RDP function on your laptop to connect to the “corporate-pc“
RDP server/workstation

\end{enumerate}
\begin{quote}

\sphinxincludegraphics[width=2.04303in,height=1.41146in]{{image4}.png}
\end{quote}
\begin{enumerate}
\item {} 
When prompted for credentials
\begin{enumerate}
\item {} 
Username: Agility

\item {} 
Password: F5Agility

\sphinxincludegraphics[width=1.48020in,height=2.12500in]{{image5}.png}

\end{enumerate}

\item {} 
Use the VMware Horizon View client to access the connection server
\begin{enumerate}
\item {} 
VMware Horizon Client

\item {} \begin{itemize}
\item {} 
New server

\end{itemize}

\end{enumerate}

\item {} 
Connection Server address “vmw-connsvr1c.demoisfun.net”

\item {} 
When prompted for credentials
\begin{enumerate}
\item {} 
Username: demo01

\item {} 
Password: password

\end{enumerate}

\item {} 
Select the View desktop (Agility—Rt click and Launch)

\item {} 
Scroll down to task bar if needed

\item {} 
Open Notepad and type in something.

\item {} 
Slide the blue RDP indicator to the left

\item {} 
Close the View client. (press the X in Agility Toolbar-was under the
RDP)

\item {} \begin{description}
\item[{Open View client and try to reconnect to “vmw-connsvr1c.}] \leavevmode
demoisfun.net”

\end{description}

\item {} 
Notepad should still be on the desktop with the test you input

\item {} 
Close the View client. (press the X in Agility Toolbar)

\item {} 
{\color{red}\bfseries{}**}Keep the RDP session open for Task 2 **

\end{enumerate}


\subsubsection{TASK 2 \textendash{} Load Balance VMware View connection servers}
\label{\detokenize{class2/module1/lab1:task-2-load-balance-vmware-view-connection-servers}}
Use the F5 iApp for VMware View to configure a load balancing
environment for the Connection Servers. This will increase the number of
Connection Servers available to internal users and load balance access
to these resources (Internal use case with F5 load balancing)

\sphinxincludegraphics[width=4.94792in,height=3.20833in]{{image6}.png}

Figure 2 - Load balance Connection Servers

\sphinxstylestrong{Deploy the iApp}
\begin{enumerate}
\item {} 
Access the F5 Config GUI from the “corporate-pc” RDP
server/workstation \textendash{}
\begin{enumerate}
\item {} 
\sphinxurl{https://f5-bigip1a.demoisfun.net} (192.168.10.216)
\begin{enumerate}
\item {} 
Username: admin

\item {} 
Password: password

\end{enumerate}

\end{enumerate}

\item {} 
Create a new Application Service
\begin{enumerate}
\item {} 
iApps \textgreater{}\textgreater{} Application Services

\item {} 
Press the \sphinxstylestrong{Create} button

\item {} 
Name the Application Service \sphinxstylestrong{VM\_LAB\_1\_LBCS}

\item {} 
Select \sphinxstylestrong{f5.vmware\_view.v1.5.1} for the template

\end{enumerate}

\end{enumerate}


\begin{savenotes}\sphinxattablestart
\centering
\begin{tabulary}{\linewidth}[t]{|T|}
\hline
\\
\hline
\end{tabulary}
\par
\sphinxattableend\end{savenotes}
\begin{enumerate}
\item {} 
Review the \sphinxstylestrong{Welcome to the iAPP template for VMware Horizon View}

\item {} 
Note the \sphinxstylestrong{Template Options} (leave these default)

\item {} 
\sphinxstylestrong{Big-IP Access Policy Manager} (Set this to \sphinxstylestrong{No} for this
exercise)

\item {} 
SSL Encryption (Certs are preloaded for this exercise)

\end{enumerate}


\begin{savenotes}\sphinxattablestart
\centering
\begin{tabulary}{\linewidth}[t]{|T|T|}
\hline
\sphinxstylethead{\sphinxstyletheadfamily 
How should the BIG-IP system handle encrypted traffic?
\unskip}\relax &\sphinxstylethead{\sphinxstyletheadfamily 
Terminate SSL for clients, re-encrypt to View servers (SSL-bridging)
\unskip}\relax \\
\hline
Which SSL certificate do you want to use?
&
wild.demoisfun.net.crt
\\
\hline
Which SSL private key do you want to use
&
wild.demoisfun.net.key
\\
\hline
\end{tabulary}
\par
\sphinxattableend\end{savenotes}
\begin{enumerate}
\item {} 
\sphinxstylestrong{PC Over IP} (leave these default \textendash{} No PCoIP connections…)

\item {} 
{\color{red}\bfseries{}**}Virtual Servers and Pools **

\end{enumerate}


\begin{savenotes}\sphinxattablestart
\centering
\begin{tabulary}{\linewidth}[t]{|T|T|}
\hline
\sphinxstylethead{\sphinxstyletheadfamily 
What virtual server IP address do you want to use for remote, untrusted clients?
\unskip}\relax &\sphinxstylethead{\sphinxstyletheadfamily 
192.168.10.150
\unskip}\relax \\
\hline
What is the associated service port?
&
443
\\
\hline
What FQDN will clients use to access the View environment
&
vmw-LB-CS.demoisfun.net
\\
\hline
Which Servers should be included in this pool
&
192.168.10.212

192.168.10.213
\\
\hline
\end{tabulary}
\par
\sphinxattableend\end{savenotes}
\begin{enumerate}
\item {} 
\sphinxstylestrong{Client Optimization} (leave these default—Do not compress…)

\item {} 
\sphinxstylestrong{Application Health}
\begin{enumerate}
\item {} 
Use the pulldown to select a standard https monitor

\end{enumerate}

\item {} 
Press the \sphinxstylestrong{Finished} button

\end{enumerate}


\paragraph{View the objects which were created by the iApp}
\label{\detokenize{class2/module1/lab1:view-the-objects-which-were-created-by-the-iapp}}\begin{enumerate}
\item {} 
Select the Components tab at the top of the page
\begin{quote}

\sphinxincludegraphics[width=3.32292in,height=1.05208in]{{image7}.png}
\end{quote}

\end{enumerate}
\begin{enumerate}
\item {} 
Is the Virtual server available?

\item {} 
Are the pool members available?

\item {} 
What is the node status? Why?

\item {} 
Note that a persistence profile was created
\begin{enumerate}
\item {} 
Check Match Across Services

\item {} 
Press update

\item {} 
Note the error at the top of the page

\end{enumerate}

\item {} 
Return to iApp\textgreater{}\textgreater{}Application Services

\item {} 
Review the remaining parameters (any questions)

\end{enumerate}


\paragraph{View the properties of the iApp}
\label{\detokenize{class2/module1/lab1:view-the-properties-of-the-iapp}}\begin{enumerate}
\item {} 
Select the Properties tab at the top of the page

\item {} 
\sphinxincludegraphics[width=3.15625in,height=1.29167in]{{image8}.png}

\item {} 
Use the pull down next to Application Service:

\item {} 
Select Advanced

\item {} 
Note the check in Strict Updates
\begin{enumerate}
\item {} 
Is this related to the screen when editing the persistence
profile?

\item {} 
What are the pro’s and con’s of unchecking this parameter?

\end{enumerate}

\end{enumerate}


\paragraph{Test the connection server load balancing using both VMware View client and browser access methods.}
\label{\detokenize{class2/module1/lab1:test-the-connection-server-load-balancing-using-both-vmware-view-client-and-browser-access-methods}}\begin{enumerate}
\item {} 
Use the RDP function on your laptop to connect to the “corporate-pc”
RDP server/workstation
\begin{enumerate}
\item {} 
Same process as Task 1 if you are not still connected

\end{enumerate}

\item {} 
Open View client and connect to the Virtual Server just created with
iApp.
\begin{enumerate}
\item {} 
+New Server
\begin{enumerate}
\item {} 
vmw-LB-CS.demoisfun.net (192.168.10.150)

\item {} 
IP address will not work—Certificate contains demoisfun.net

\end{enumerate}

\end{enumerate}

\item {} 
When prompted for credentials
\begin{enumerate}
\item {} 
Username: demo01

\item {} 
Password: password

\end{enumerate}

\item {} 
Select the View desktop (Agility)

\item {} 
Use connect button to access

\item {} 
Slide the blue RDP indicator to the left

\item {} 
Close the View client. (press the X in Agility Toolbar-was under the
RDP)

\item {} 
Use a supported browser to access the VDI (IE on the RDP
workstation)

\sphinxincludegraphics[width=4.37500in,height=1.28125in]{{image9}.png}

\item {} 
\sphinxurl{https://vmw-LB-CS.demoisfun.net}

\item {} 
Select VMware Horizon View HTML access

\item {} 
Log in
\begin{enumerate}
\item {} 
Username: demo01

\item {} 
Password: password

\end{enumerate}

\item {} 
Select (Agility)

\item {} 
Accept Cert Warnings

\item {} 
Verify that the desktop functions

\item {} 
Close the browser window

\end{enumerate}


\subsubsection{TASK 3 \textendash{} Access View Desktop environment through Security Server}
\label{\detokenize{class2/module1/lab1:task-3-access-view-desktop-environment-through-security-server}}
Test the functional VMware View environment using external Security
Servers. (External use case without F5 integration)

Note: This environment shows a user connecting to a native VMware
security server which is statically mapped to a VMware connection
server. This is a non-redundant external access model

\sphinxincludegraphics[width=5.25000in,height=3.18750in]{{image10}.png}

Figure 3 - Access external View Desktop

Access the VDI using the Security Server from a Windows Server RDP
session
\begin{enumerate}
\item {} 
Use the RDP function on your laptop to connect to the
“\sphinxstylestrong{home-pc}” RDP server/workstation

\end{enumerate}

\sphinxincludegraphics[width=2.04236in,height=1.41111in]{{image4}.png}
\begin{enumerate}
\item {} 
When prompted for credentials
\begin{enumerate}
\item {} 
Username: agility

\item {} 
Password: F5Agility

\sphinxincludegraphics[width=1.32738in,height=2.22370in]{{image11}.png}

\end{enumerate}

\item {} 
Use the VMware Horizon View client to access the security server
\begin{enumerate}
\item {} 
+New Server

\item {} 
Security Server address “vmw-secursvr1a.demoisfun.net”

\end{enumerate}

\item {} 
When prompted for credentials
\begin{enumerate}
\item {} 
Username: demo01

\item {} 
Password: password

\end{enumerate}

\item {} 
Select the View desktop (Right Click on Agility - Launch)

\item {} 
Slide the blue RDP indicator to the left

\item {} 
Close the View client. (press the X in Agility Toolbar-was under the
RD)
\begin{enumerate}
\item {} 
vmw-secursvr1a.demoisfun.net

\end{enumerate}

\item {} 
Use a supported browser to access the VDI (IE on the RDP
workstation)

\sphinxincludegraphics[width=4.37500in,height=1.28125in]{{image9}.png}

\item {} 
Access the application through your browser \sphinxurl{https://}
vmw-secursvr1a.demoisfun.net
\begin{enumerate}
\item {} 
vmw-secursvr1a.demoisfun.net

\item {} 
Username: demo01

\item {} 
Password: password

\end{enumerate}

\item {} 
Select VMware Horizon View HTML access

\item {} 
Log in
\begin{enumerate}
\item {} 
Username: demo01

\item {} 
Password: password

\end{enumerate}

\item {} 
Select (Agility)

\item {} 
Accept Cert at warning

\item {} 
Select (Agility)

\item {} 
Verify that the desktop functions
\begin{enumerate}
\item {} 
Scroll down to taskbar

\end{enumerate}

\item {} 
Close the browser

\end{enumerate}

192.168.3.150


\subsubsection{TASK 4 \textendash{} Load Balance VMware View security servers}
\label{\detokenize{class2/module1/lab1:task-4-load-balance-vmware-view-security-servers}}
Use the F5 iApp for VMware View to configure a load balancing
environment for the Security Servers. This will increase the number of
Security Servers available to internal users and load balance access to
these resources (External use case with F5 load balancing)

Note: This environment load balances 2 external facing Security Servers.
These Security Servers are directly mapped to 2 existing connection
servers in the environment (not the 2 Connections Servers that are load
balances in the steps above)

\sphinxincludegraphics[width=4.63542in,height=3.06250in]{{image12}.png}

Figure 4 - Load balance Security Servers

\sphinxstylestrong{Deploy the iApp}
\begin{enumerate}
\item {} 
Use the RDP function on your laptop to connect to the “corporate-pc”
RDP server/workstation
\begin{enumerate}
\item {} 
Same process as Task 1 if you are not still connected

\end{enumerate}

\item {} 
Create a new Application Service by selecting
\begin{enumerate}
\item {} 
iApps \textgreater{}\textgreater{} Application Services

\item {} 
Press the \sphinxstylestrong{Create} button

\item {} 
Name the Application Service \sphinxstylestrong{VM\_LAB\_1\_LBSS}

\item {} 
Select \sphinxstylestrong{f5.vmware\_view.v1.5.1} for the template

\end{enumerate}

\end{enumerate}


\begin{savenotes}\sphinxattablestart
\centering
\begin{tabulary}{\linewidth}[t]{|T|}
\hline
\\
\hline
\end{tabulary}
\par
\sphinxattableend\end{savenotes}
\begin{enumerate}
\item {} 
Review the \sphinxstylestrong{Welcome to the iAPP template for VMware Horizon View}

\item {} 
Note the \sphinxstylestrong{Template Options} (leave these default)

\item {} 
\sphinxstylestrong{Big-IP Access Policy Manager} (Set this to \sphinxstylestrong{No} for this
exercise)

\item {} 
\sphinxstylestrong{SSL Encryption} (Certs are preloaded for this exercise)

\end{enumerate}


\begin{savenotes}\sphinxattablestart
\centering
\begin{tabulary}{\linewidth}[t]{|T|T|}
\hline
\sphinxstylethead{\sphinxstyletheadfamily 
How should the BIG-IP system handle encrypted traffic?
\unskip}\relax &\sphinxstylethead{\sphinxstyletheadfamily 
Terminate SSL for clients, re-encrypt…\sphinxstylestrong{(SSL-Bridging)}
\unskip}\relax \\
\hline
Which SSL certificate do you want to use?
&
wild.demoisfun.net.crt
\\
\hline
Which SSL private key do you want to use?
&
wild.demoisfun.net.key
\\
\hline
\end{tabulary}
\par
\sphinxattableend\end{savenotes}
\begin{enumerate}
\item {} 
\sphinxstylestrong{PC Over IP} (leave these default \textendash{} No PCoIP connections…)

\item {} 
{\color{red}\bfseries{}**}Virtual Servers and Pools **

\end{enumerate}


\begin{savenotes}\sphinxattablestart
\centering
\begin{tabulary}{\linewidth}[t]{|T|T|}
\hline
\sphinxstylethead{\sphinxstyletheadfamily 
What virtual server IP address do you want to use for remote, untrusted clients?
\unskip}\relax &\sphinxstylethead{\sphinxstyletheadfamily 
192.168.3.150
\unskip}\relax \\
\hline
What is the associated service port?
&
443
\\
\hline
What FQDN will clients use to access the View environment?
&
vmw-LB-SS.demoisfun.net
\\
\hline
Which Servers should be included in this pool?
&
192.168.3.214

192.168.3.215
\\
\hline
\end{tabulary}
\par
\sphinxattableend\end{savenotes}
\begin{enumerate}
\item {} 
\sphinxstylestrong{Client Optimization} (leave these default—Do not compress…)

\item {} 
\sphinxstylestrong{Application Health}
\begin{enumerate}
\item {} 
Use the pulldown to select a standard https monitor

\end{enumerate}

\item {} 
Press the \sphinxstylestrong{Finished} button

\end{enumerate}


\paragraph{View the objects which were created by the iApp}
\label{\detokenize{class2/module1/lab1:id7}}\begin{enumerate}
\item {} 
Select the Components tab at the top of the page

\item {} 
Is the Virtual server available?

\item {} 
Are the pool members available?

\item {} 
Is the Node Available?

\item {} 
Review the remaining parameters (any questions)

\end{enumerate}


\paragraph{Test the Security Server load balancing using both VMware View client and browser access methods}
\label{\detokenize{class2/module1/lab1:test-the-security-server-load-balancing-using-both-vmware-view-client-and-browser-access-methods}}\begin{enumerate}
\item {} 
Use the RDP function on your laptop to connect to the “home-pc” RDP
server/workstation

\item {} 
Open View client and connect to the Virtual Server just created with
iApp.
\begin{enumerate}
\item {} 
+New Server
\begin{enumerate}
\item {} 
vmw-LB-SS.demoisfun.net (192.168.3.150)

\item {} 
IP address will not work—Certificate contains demoisfun.net

\end{enumerate}

\end{enumerate}

\item {} 
When prompted for credentials
\begin{enumerate}
\item {} 
Username: demo01

\item {} 
Password: password

\end{enumerate}

\item {} 
Select the View desktop (Agility)

\item {} 
Use connect button to access

\item {} 
Slide the blue RDP indicator to the left

\item {} 
Close the View client. (press the X in Agility Toolbar-was under the
RD)

\item {} 
Use a supported browser to access the VDI (IE on the RDP
workstation)

\sphinxincludegraphics[width=4.37500in,height=1.28125in]{{image9}.png}

\item {} 
\sphinxurl{https://vmw-LB-SS.demoisfun.net}

\item {} 
Select VMware Horizon View HTML access

\item {} 
Enter Credentials
\begin{enumerate}
\item {} 
Username: demo01

\item {} 
Password: password

\end{enumerate}

\item {} 
Select (Agility)

\item {} 
Accept Cert warning

\item {} 
Select (Agility)

\item {} 
Verify that the desktop functions

\item {} 
Close the browser

\end{enumerate}


\subsubsection{TASK 5 \textendash{} Replace Security Servers and leverage APM as a PCOIP proxy}
\label{\detokenize{class2/module1/lab1:task-5-replace-security-servers-and-leverage-apm-as-a-pcoip-proxy}}
\sphinxstylestrong{Use the VMware View iApp to replace Security Server to proxy PCoIP
traffic}

Note: This environment will utilize Big-IP as a PCOIP Proxy. This
eliminates the requirement for all Security Servers. The Connection
Servers will be load balanced. Authentication is handled by the F5 APM
module

\sphinxincludegraphics[width=5.67708in,height=3.35417in]{{image13}.png}

Figure 5 - Replace Security Servers

\sphinxstylestrong{Deploy the iApp}
\begin{enumerate}
\item {} 
Use the RDP function on your laptop to connect to the “corporate-pc”
RDP server/workstation
\begin{enumerate}
\item {} 
Same process as Task 1 if you are not still connected

\end{enumerate}

\item {} 
Create a new Application Service by selecting iApps -\textgreater{} Application
Services and selecting Create
\begin{enumerate}
\item {} 
iApps \textgreater{}\textgreater{} Application Services

\item {} 
Press the \sphinxstylestrong{Create} button

\item {} 
Name the Application Service \sphinxstylestrong{VM\_LAB\_1\_PCOIP}

\item {} 
Select \sphinxstylestrong{f5.vmware\_view.v1.5.1} for the template

\end{enumerate}

\end{enumerate}


\begin{savenotes}\sphinxattablestart
\centering
\begin{tabulary}{\linewidth}[t]{|T|}
\hline
\\
\hline
\end{tabulary}
\par
\sphinxattableend\end{savenotes}


\paragraph{iApp Configuration}
\label{\detokenize{class2/module1/lab1:iapp-configuration}}\begin{enumerate}
\item {} 
Review the \sphinxstylestrong{Welcome to the iAPP template for VMware Horizon View}

\item {} 
Note the \sphinxstylestrong{Template Options} (leave these default)

\item {} 
\sphinxstylestrong{Big-IP Access Policy Manager}

\end{enumerate}


\begin{savenotes}\sphinxattablestart
\centering
\begin{tabulary}{\linewidth}[t]{|T|T|}
\hline
\sphinxstylethead{\sphinxstyletheadfamily 
Do you want to deploy BIG-IP Access Policy Manager?
\unskip}\relax &\sphinxstylethead{\sphinxstyletheadfamily 
Yes, deploy BIG-IP Access Policy Manager
\unskip}\relax \\
\hline&\\
\hline
Do you want to support browser based connections, including the View HTML5 client?
&
Yes, support HTML 5 view clientless browser connections
\\
\hline
Should the BIG-IP system support RSA SecureID two-factor authentication
&
NO, do not support RSA SecureID two-factor authentication
\\
\hline
Should the BIG\_IP system show a message to View users during logon
&
No, do not add a message during logon
\\
\hline
What is the NetBIOS domain name for your environment
&
demoisfun
\\
\hline
Create a new AAA Server object {\color{red}\bfseries{}**}or select an existing one **
&
AD1
\\
\hline
\end{tabulary}
\par
\sphinxattableend\end{savenotes}
\begin{enumerate}
\item {} 
SSL Encryption (Certs are preloaded for this exercise)

\end{enumerate}


\begin{savenotes}\sphinxattablestart
\centering
\begin{tabulary}{\linewidth}[t]{|T|T|}
\hline
\sphinxstylethead{\sphinxstyletheadfamily 
How should the BIG-IP system handle encrypted traffic?
\unskip}\relax &\sphinxstylethead{\sphinxstyletheadfamily 
Terminate SSL for clients, re-encrypt…\sphinxstylestrong{(SSL-Bridging)}
\unskip}\relax \\
\hline
Which SSL certificate do you want to use?
&
wild.demoisfun.net.crt
\\
\hline
Which SSL private key do you want to use?
&
wild.demoisfun.net.key
\\
\hline
\end{tabulary}
\par
\sphinxattableend\end{savenotes}
\begin{enumerate}
\item {} 
\sphinxstylestrong{PC Over IP} (leave these default)

\item {} 
{\color{red}\bfseries{}**}Virtual Servers and Pools **

\end{enumerate}


\begin{savenotes}\sphinxattablestart
\centering
\begin{tabulary}{\linewidth}[t]{|T|T|}
\hline
\sphinxstylethead{\sphinxstyletheadfamily 
What virtual server IP address do you want to use for remote, untrusted clients?
\unskip}\relax &\sphinxstylethead{\sphinxstyletheadfamily 
192.168.3.152
\unskip}\relax \\
\hline
What is the associated service port?
&
443
\\
\hline
What FQDN will clients use to access the View environment?
&
vmw-PROXY-VIEW.demoisfun.net
\\
\hline
Which Servers should be included in this pool?
&
192.168.10.212

192.168.10.213
\\
\hline
\end{tabulary}
\par
\sphinxattableend\end{savenotes}
\begin{enumerate}
\item {} 
\sphinxstylestrong{Application Health}
\begin{enumerate}
\item {} 
Use the pull down to select a standard https monitor

\end{enumerate}

\item {} 
Press the \sphinxstylestrong{Finished} button

\end{enumerate}


\paragraph{View the objects which were created by the iApp}
\label{\detokenize{class2/module1/lab1:id12}}\begin{enumerate}
\item {} 
Select the Components tab at the top of the page

\item {} 
Note the increase in objects compared to Task 2 and Task 4

\item {} 
Are the pool members available?

\item {} 
Note the APM objects which were not present in the prior exercises

\item {} 
Review the remaining parameters (any questions)

\end{enumerate}


\paragraph{Test the APM (PCoIP) functionality using both VMware View client and browser access methods}
\label{\detokenize{class2/module1/lab1:test-the-apm-pcoip-functionality-using-both-vmware-view-client-and-browser-access-methods}}
Use the RDP function on your laptop to connect to the “home-pc” or use
the browser / local view client on your laptop to access
vmw-PROXY-VIEW.demoisfun.net
\begin{enumerate}
\item {} 
Open View client and connect to the Virtual Server just created with
iApp.
\begin{enumerate}
\item {} 
vmw-PROXY-VIEW.demoisfun.net (192.168.3.152)

\item {} 
IP address will not work—Certificate contains demoisfun.net

\end{enumerate}

\item {} 
When prompted for credentials
\begin{enumerate}
\item {} 
Username: demo01

\item {} 
Password: password

\end{enumerate}

\item {} 
If authentication fails
\begin{enumerate}
\item {} 
Access Policy\textgreater{}\textgreater{}Manage Sessions

\item {} 
Look at the entire session log
\begin{enumerate}
\item {} 
More detail can be captured by enabling debug

\end{enumerate}

\item {} 
Note the clock skew error

\item {} 
Use the “Corporate PC” to Connect to the F5 Big IP GUI
\sphinxurl{https://192.168.10.216}

\item {} 
Set the time on the big IP to match the time on the corporate-pc
\begin{enumerate}
\item {} 
date MMDDhhmm Keep in mind—the big IP uses military time 1:25
PM = 13:25

\end{enumerate}

\item {} 
Return to step 1

\end{enumerate}

\item {} 
Select the View desktop (Agility)

\item {} 
Use connect button to access

\item {} 
Close the View client. (press the X in the upper right corner of the
screen)

\item {} 
\sphinxurl{https://192.168.3.152}
\begin{enumerate}
\item {} 
Username: demo01

\item {} 
Password: password

\end{enumerate}

\item {} 
Select (Agility) from the webtop

\item {} 
Select VMware View Client on the desktop

\item {} 
Note the error and inspect the certificate

\item {} 
Close the error box and cert view boxes

\item {} 
Open VMware View Client
\begin{enumerate}
\item {} 
\sphinxhref{https://vmw-PROXY-VIEW.demoisfun.net}{vmw-PROXY-VIEW.demoisfun.net}

\item {} 
Username:demo01

\item {} 
Password: password

\end{enumerate}

\item {} 
Select (Agility) from the webtop

\item {} 
Select VMware View client

\item {} 
When the desktop opens, open Notepad and enter some text (leave this
on the screen)

\item {} 
Slide the blue RDP indicator to the left

\item {} 
Close the View client. (press the X in Agility Toolbar-was under the
RD)

\item {} 
Use a supported browser to access the VDI (IE on the RDP
workstation)

\item {} 
\sphinxurl{https://vmw-PROXY-VIEW.demoisfun.net}

\item {} 
Select VMware Horizon View HTML access

\item {} 
Enter Credentials
\begin{enumerate}
\item {} 
Username: demo01

\item {} 
Password: password

\end{enumerate}

\item {} 
Select (Agility)

\item {} 
Select HTML5 Client

\item {} 
Verify that the desktop functions

\item {} 
Close the browser

\end{enumerate}


\section{Module 2 - Solutions for Citrix XenDesktop}
\label{\detokenize{class2/module2/module2::doc}}\label{\detokenize{class2/module2/module2:module-2-solutions-for-citrix-xendesktop}}
The purpose of this module is to build out 2 common F5 deployment with
XenDesktop.

Note: The connectivity in this environment is slower than a typical
production environment—please be patient


\subsection{Lab 2 \textendash{} Solutions for Citrix XenDesktop}
\label{\detokenize{class2/module2/lab1:lab-2-solutions-for-citrix-xendesktop}}\label{\detokenize{class2/module2/lab1::doc}}
The purpose of this lab is to build out 2 common F5 deployment with
XenDesktop.

Note: The connectivity in this environment is slower than a typical
production environment—please be patient


\subsubsection{TASK 1 \textendash{} Access XenDesktop without F5}
\label{\detokenize{class2/module2/lab1:task-1-access-xendesktop-without-f5}}
\sphinxincludegraphics[width=5.14583in,height=3.45833in]{{image14}.png}
\begin{enumerate}
\item {} 
From “corporate-pc”, use IE and browse to Citrix Storefront at
\sphinxurl{http://ctx-sf1a.demoisfun.net/Citrix/AgilityStoreWeb/}

\item {} 
When prompted for credentials
\begin{enumerate}
\item {} 
Username: demoisfun\textbackslash{}demo01

\item {} 
Password: password

\end{enumerate}

\item {} 
Right Click “Agility” and select “Start” icon to launch XenDesktop.
\begin{enumerate}
\item {} 
\sphinxstyleemphasis{Note: This takes a long time due to the Ravello implementation}

\end{enumerate}

\item {} 
Citrix “Desktop Viewer” launches and connects to XenDesktop.

\item {} 
When the windows Activation Screen Pops up..Press Cancel (windows was
not activated due to external connectivity limitations)

\item {} 
Log off using the windows start icon in the lower left corner

\item {} 
Log off the Citrix receiver client using the 01 Demo pulldown in the
upper right corner

\item {} 
Close the browser Window

\end{enumerate}


\subsubsection{TASK 2 \textendash{} Load Balance StoreFront}
\label{\detokenize{class2/module2/lab1:task-2-load-balance-storefront}}
\sphinxincludegraphics[width=5.30208in,height=2.98958in]{{image15}.png}

\sphinxstylestrong{Deploy the iApp}
\begin{enumerate}
\item {} 
Use the RDP function on your laptop to connect to the “corporate-pc”

\item {} 
Create a new Application Service by selecting iApps -\textgreater{} Application
Services and selecting Create
\begin{enumerate}
\item {} 
iApps \textgreater{}\textgreater{} Application Services

\item {} 
Press the \sphinxstylestrong{Create} button

\item {} 
Name the Application Service \sphinxstylestrong{VM\_LAB\_2\_LBSF}

\item {} 
Select \sphinxstylestrong{f5.citrix\_vdi.v2.3.0} for the template

\end{enumerate}

\end{enumerate}


\begin{savenotes}\sphinxattablestart
\centering
\begin{tabulary}{\linewidth}[t]{|T|}
\hline
\\
\hline
\end{tabulary}
\par
\sphinxattableend\end{savenotes}


\paragraph{iApp Configuration}
\label{\detokenize{class2/module2/lab1:iapp-configuration}}\begin{enumerate}
\item {} 
Review the \sphinxstylestrong{Welcome to the iApp template for XenDesktop and XenApp}

\item {} 
\sphinxstylestrong{General}

\end{enumerate}


\begin{savenotes}\sphinxattablestart
\centering
\begin{tabulary}{\linewidth}[t]{|T|T|}
\hline
\sphinxstylethead{\sphinxstyletheadfamily 
Use APM to securely proxy application (ICA) traffic and authenticate users into your Citrix environment?
\unskip}\relax &\sphinxstylethead{\sphinxstyletheadfamily 
Yes, Proxy ICA traffic and authenticate users with BIG\_IP
\unskip}\relax \\
\hline
What is the Active Directory NetBIOS Domain Name used for your Citrix servers?
&
demoisfun
\\
\hline
\end{tabulary}
\par
\sphinxattableend\end{savenotes}
\begin{enumerate}
\item {} 
\sphinxstylestrong{BIG-IP Access Policy Manager}

\end{enumerate}


\begin{savenotes}\sphinxattablestart
\centering
\begin{tabulary}{\linewidth}[t]{|T|T|}
\hline
\sphinxstylethead{\sphinxstyletheadfamily 
Do you want to replace Citrix Web Interface or StoreFront servers with the BIG-IP system?
\unskip}\relax &\sphinxstylethead{\sphinxstyletheadfamily 
“No, do not replace…”
\unskip}\relax \\
\hline
Create a new AAA object or select an existing one?
&
AD1
\\
\hline
\end{tabulary}
\par
\sphinxattableend\end{savenotes}
\begin{enumerate}
\item {} 
{\color{red}\bfseries{}**}Virtual Server for Web Interface or StoreFront servers **

\end{enumerate}


\begin{savenotes}\sphinxattablestart
\centering
\begin{tabulary}{\linewidth}[t]{|T|T|}
\hline
\sphinxstylethead{\sphinxstyletheadfamily 
How should the BIG-IP system handle encrypted traffic to Web Interface or StoreFront servers?
\unskip}\relax &\sphinxstylethead{\sphinxstyletheadfamily 
Terminate SSL for Clients, Plaintext to Citrix servers \sphinxstylestrong{(SSL offload)}
\unskip}\relax \\
\hline
Which SSL certificate do you want to use?
&
wild.demoisfun.net.crt
\\
\hline
Which SSL private key do you want to use?
&
wild.demoisfun.net.key
\\
\hline
What IP address will clients use to access the Web Interface or StoreFront servers, or the F5 Webtop?
&
192.168.3.160
\\
\hline
Did you deploy Citrix StoreFront?
&
Yes,…StoreFront 3.0 or 3.6

Note: we are running SF 3.9
\\
\hline
What is the URI used on StoreFront or Web Interface servers for XenApp or XenDesktop?
&
/Citrix/AgilityStoreWeb/

\sphinxstyleemphasis{Note that this is the same URL used to access citrix directly in Task 1}
\\
\hline
\end{tabulary}
\par
\sphinxattableend\end{savenotes}
\begin{enumerate}
\item {} 
\sphinxstylestrong{Web Interface or StoreFront servers}

\end{enumerate}


\begin{savenotes}\sphinxattablestart
\centering
\begin{tabulary}{\linewidth}[t]{|T|T|}
\hline
\sphinxstylethead{\sphinxstyletheadfamily 
What DNS name will clients use to reach the Web Interface or StoreFront servers?
\unskip}\relax &\sphinxstylethead{\sphinxstyletheadfamily 
ctx-LB-SF.demoisfun.net
\unskip}\relax \\
\hline
Which port have you configured for Web Interface or StoreFront HTTP traffic?
&
80
\\
\hline
What are the IP addresses of your Web Interface or StoreFront servers?
&
192.168.10.220

192.168.10.221
\\
\hline
Which Monitor do you want to use
&
http
\\
\hline
\end{tabulary}
\par
\sphinxattableend\end{savenotes}
\begin{enumerate}
\item {} 
\sphinxstylestrong{Virtual Server for XML Broker or Desktop Delivery Controller (DDC)
Servers}

\end{enumerate}


\begin{savenotes}\sphinxattablestart
\centering
\begin{tabulary}{\linewidth}[t]{|T|T|}
\hline
\sphinxstylethead{\sphinxstyletheadfamily 
What IP address do you want to use for the XML Broker or DDC farm virtual server?
\unskip}\relax &\sphinxstylethead{\sphinxstyletheadfamily 
192.168.10.161
\unskip}\relax \\
\hline
How will requests from the Web Interface or StoreFront servers arrive?
&
XML Broker or DCC requests will arrive unencrypted (HTTP)
\\
\hline
\end{tabulary}
\par
\sphinxattableend\end{savenotes}
\begin{enumerate}
\item {} 
{\color{red}\bfseries{}**}XML Broker or DDC Servers **

\end{enumerate}


\begin{savenotes}\sphinxattablestart
\centering
\begin{tabulary}{\linewidth}[t]{|T|T|}
\hline
\sphinxstylethead{\sphinxstyletheadfamily 
What are the IP addresses of your XML Broker or DDC servers?
\unskip}\relax &\sphinxstylethead{\sphinxstyletheadfamily 
192.168.10.222

192.168.10.223
\unskip}\relax \\
\hline
Which monitor do you want to use?
&
http
\\
\hline
\end{tabulary}
\par
\sphinxattableend\end{savenotes}
\begin{enumerate}
\item {} 
Press the \sphinxstylestrong{Finished} button

\end{enumerate}


\paragraph{Test connectivity}
\label{\detokenize{class2/module2/lab1:test-connectivity}}\begin{enumerate}
\item {} 
Use the RDP function on your laptop to connect to the “home-pc”

\item {} 
Launch IE and browse to \sphinxurl{http://ctx-lb-sf.demoisfun.net}

\item {} 
When prompted for credentials
\begin{enumerate}
\item {} 
Username: demo01

\item {} 
Password: password

\end{enumerate}

\item {} 
Storefront is displayed with Agility icon.

\item {} 
Right Click on “Agility” and select “Start” icon to launch
XenDesktop.
\begin{enumerate}
\item {} 
\sphinxstyleemphasis{Note: This takes a long time due to the Ravello implementation}

\end{enumerate}

\item {} 
Log off using the windows start icon in the lower left corner

\item {} 
Log off the Citrix receiver client using the 01 Demo pulldown in the
upper right corner

\item {} 
Close the browser Window

\end{enumerate}


\subsubsection{TASK 3 \textendash{} Reconfigure the iApp to Replace StoreFront}
\label{\detokenize{class2/module2/lab1:task-3-reconfigure-the-iapp-to-replace-storefront}}
\sphinxincludegraphics[width=5.39583in,height=3.21875in]{{image16}.png}

\sphinxstylestrong{Deploy the iApp}
\begin{enumerate}
\item {} 
Use the RDP function on your laptop to connect to the “corporate-pc”

\item {} 
Create a new Application Service by selecting iApps -\textgreater{} Application
Services and selecting Create
\begin{enumerate}
\item {} 
iApps \textgreater{}\textgreater{} Application Services

\item {} 
Click on \sphinxstylestrong{VM\_LAB\_2\_LBSF}

\item {} 
Click the \sphinxstylestrong{Reconfigure} link near the top

\end{enumerate}

\end{enumerate}


\begin{savenotes}\sphinxattablestart
\centering
\begin{tabulary}{\linewidth}[t]{|T|}
\hline
\\
\hline
\end{tabulary}
\par
\sphinxattableend\end{savenotes}


\paragraph{iApp Configuration}
\label{\detokenize{class2/module2/lab1:id5}}\begin{enumerate}
\item {} 
\sphinxstylestrong{BIG-IP Access Policy Manager}

\end{enumerate}


\begin{savenotes}\sphinxattablestart
\centering
\begin{tabulary}{\linewidth}[t]{|T|T|}
\hline
\sphinxstylethead{\sphinxstyletheadfamily 
Do you want to replace Citrix Web Interface or StoreFront servers with the BIG-IP system?
\unskip}\relax &\sphinxstylethead{\sphinxstyletheadfamily 
“Yes, replace Citrix…”
\unskip}\relax \\
\hline&\\
\hline
\end{tabulary}
\par
\sphinxattableend\end{savenotes}
\begin{enumerate}
\item {} 
Scroll through the template and note that the storefront pool members
are no longer present

\item {} 
Press the \sphinxstylestrong{Finished} button

\end{enumerate}


\paragraph{Test connectivity}
\label{\detokenize{class2/module2/lab1:id6}}\begin{enumerate}
\item {} 
Use the RDP function on your laptop to connect to the “home-pc”

\item {} 
Launch IE and browse to \sphinxurl{http://ctx-lb-sf.demoisfun.net}

\item {} 
When prompted for credentials
\begin{enumerate}
\item {} 
Username: demo01

\item {} 
Password: password

\end{enumerate}

\item {} 
APM webtop is displayed with Agility icon.

\item {} 
Click on Agility icon to launch XenDesktop.

\item {} 
When the windows Activation Screen Pops up..Press Cancel (windows was
not activated due to external connectivity limitations)

\item {} 
Log off using the windows start icon in the lower left
corner\sphinxincludegraphics[width=4.53125in,height=2.03125in]{{image17}.png}

\item {} 
Logout using the Logout button in the upper right corner of the
screen

\item {} 
Close the broiwser window

\end{enumerate}


\section{Module 3 - Microsoft RDS Proxy}
\label{\detokenize{class2/module3/module3::doc}}\label{\detokenize{class2/module3/module3:module-3-microsoft-rds-proxy}}
The purpose of this module is access an internal RDS server from an
external client.


\subsection{Lab 3 \textendash{} Microsoft RDS proxy}
\label{\detokenize{class2/module3/lab1::doc}}\label{\detokenize{class2/module3/lab1:lab-3-microsoft-rds-proxy}}
The purpose of this lab is access an internal RDS server from an
external client.


\subsubsection{TASK 1 \textendash{} Access Terminal Server from external}
\label{\detokenize{class2/module3/lab1:task-1-access-terminal-server-from-external}}
\sphinxincludegraphics[width=5.58333in,height=2.96875in]{{image18}.png}

\sphinxstylestrong{Deploy the iApp}
\begin{enumerate}
\item {} 
Use the RDP function on your laptop to connect to the “corporate-pc”

\item {} 
Connect to the F5 config GUI
\begin{enumerate}
\item {} 
\sphinxurl{https://192.168.10.216}

\item {} 
Username: admin

\item {} 
Password: password

\end{enumerate}

\item {} 
Create an NTLM Machine Account
\begin{enumerate}
\item {} 
Access \textgreater{}\textgreater{}Authentication\textgreater{}\textgreater{}NTLM\textgreater{}\textgreater{}Machine Account

\end{enumerate}

\end{enumerate}


\begin{savenotes}\sphinxattablestart
\centering
\begin{tabulary}{\linewidth}[t]{|T|T|}
\hline
\sphinxstylethead{\sphinxstyletheadfamily 
Name
\unskip}\relax &\sphinxstylethead{\sphinxstyletheadfamily 
AD1-f5-bigip1a
\unskip}\relax \\
\hline
Machine Account Name
&
f5-bigip1a
\\
\hline
Domain FQDN
&
demoisfun.net
\\
\hline
Domain Controller FQDN
&
dif-ad1.demoisfun.net
\\
\hline
Admin User
&
administrator
\\
\hline
Password
&
password
\\
\hline
\end{tabulary}
\par
\sphinxattableend\end{savenotes}
\begin{enumerate}
\item {} 
Us the \sphinxstylestrong{JOIN} button to create the machine account

\item {} 
Create a new Application Service by selecting iApps -\textgreater{} Application
Services and selecting Create
\begin{enumerate}
\item {} 
iApps \textgreater{}\textgreater{} Application Services

\item {} 
Press the \sphinxstylestrong{Create} button

\item {} 
Name the Application Service \sphinxstylestrong{VM\_LAB\_3\_RDS}

\item {} 
Select \sphinxstylestrong{f5.microsoft\_rds\_remote\_access.v1.0.0} for the
template

\end{enumerate}

\end{enumerate}


\begin{savenotes}\sphinxattablestart
\centering
\begin{tabulary}{\linewidth}[t]{|T|}
\hline
\\
\hline
\end{tabulary}
\par
\sphinxattableend\end{savenotes}


\paragraph{iApp Configuration}
\label{\detokenize{class2/module3/lab1:iapp-configuration}}\begin{enumerate}
\item {} 
Review the \sphinxstylestrong{Welcome to the iApp template for Remote Desktop
Gateway}

\item {} 
\sphinxstylestrong{Template Options}

\end{enumerate}


\begin{savenotes}\sphinxattablestart
\centering
\begin{tabulary}{\linewidth}[t]{|T|T|}
\hline
\sphinxstylethead{\sphinxstyletheadfamily 
Do you want to deploy BIG-IP APM as an RDP proxy?
\unskip}\relax &\sphinxstylethead{\sphinxstyletheadfamily 
Yes, deploy BIG-IP Access Policy Manager
\unskip}\relax \\
\hline&\\
\hline
\end{tabulary}
\par
\sphinxattableend\end{savenotes}
\begin{enumerate}
\item {} 
\sphinxstylestrong{Access Policy Manager}

\end{enumerate}


\begin{savenotes}\sphinxattablestart
\centering
\begin{tabulary}{\linewidth}[t]{|T|T|}
\hline
\sphinxstylethead{\sphinxstyletheadfamily 
Do you want to create a new AAA server, or use an existing AAA server?
\unskip}\relax &\sphinxstylethead{\sphinxstyletheadfamily 
AD1
\unskip}\relax \\
\hline
Which NTLM machine account should be used for Kerberos delegation?
&
AD1-f5-bigip1a
\\
\hline
\end{tabulary}
\par
\sphinxattableend\end{savenotes}
\begin{enumerate}
\item {} 
\sphinxstylestrong{Network (leave defaults)}

\item {} 
\sphinxstylestrong{SSL Encryption}

\end{enumerate}


\begin{savenotes}\sphinxattablestart
\centering
\begin{tabulary}{\linewidth}[t]{|T|T|}
\hline
\sphinxstylethead{\sphinxstyletheadfamily 
Which SSL certificate do you want to use?
\unskip}\relax &\sphinxstylethead{\sphinxstyletheadfamily 
wild.demoisfun.net.crt
\unskip}\relax \\
\hline
Which SSL private key do you want to use?
&
wild.demoisfun.net.key
\\
\hline
\end{tabulary}
\par
\sphinxattableend\end{savenotes}
\begin{enumerate}
\item {} 
{\color{red}\bfseries{}**}Virtual Servers and Pools **

\end{enumerate}


\begin{savenotes}\sphinxattablestart
\centering
\begin{tabulary}{\linewidth}[t]{|T|T|}
\hline
\sphinxstylethead{\sphinxstyletheadfamily 
What IP address do you want to use for the virtual server(s)?
\unskip}\relax &\sphinxstylethead{\sphinxstyletheadfamily 
192.168.3.156
\unskip}\relax \\
\hline
How would you like to secure your hosts?
&
Allow any host
\\
\hline
\end{tabulary}
\par
\sphinxattableend\end{savenotes}
\begin{enumerate}
\item {} 
Press the \sphinxstylestrong{Finished} button

\end{enumerate}


\paragraph{Test the RDS proxy functionality using RDS Client}
\label{\detokenize{class2/module3/lab1:test-the-rds-proxy-functionality-using-rds-client}}\begin{enumerate}
\item {} 
Use the RDP function on your laptop to connect to the “home-pc”

\item {} 
Launch RDS client (on desktop).
\begin{enumerate}
\item {} 
Select the “Show Options” Pulldown
\begin{enumerate}
\item {} 
Select the “Advanced” tab
\begin{enumerate}
\item {} 
Select the Settings button

\item {} 
Note the configuration of the RD Gateway.
msft-proxy-rds.demoisfun.net resolves to the address
192.168.3.156 which was configured in the iApp

\end{enumerate}

\end{enumerate}

\end{enumerate}
\begin{quote}

\sphinxincludegraphics[width=2.05729in,height=2.31385in]{{image19}.png}
\end{quote}

\end{enumerate}
\begin{enumerate}
\item {} 
Verify the settings and click the OK button

\end{enumerate}
\begin{enumerate}
\item {} 
Under “General” tab, in the “Computer” field, type in the name of
the host you want to RDP to which is “dif-termsvr.demoisfun.net”

\sphinxincludegraphics[width=2.06771in,height=2.38695in]{{image20}.png}

\item {} 
Clock “Save”

\item {} 
Click “Connect”

\end{enumerate}
\begin{enumerate}
\item {} 
When prompted for credentials
\begin{enumerate}
\item {} 
Username: demo01

\item {} 
Password: password

\end{enumerate}

\item {} 
Accept Certificate warning
\begin{quote}

\sphinxincludegraphics[width=1.82813in,height=1.68013in]{{image21}.png}
\end{quote}

\end{enumerate}
\begin{enumerate}
\item {} 
You are connected to dif-termsvr.demoisfun.net

\end{enumerate}
\begin{enumerate}
\item {} 
Use the “Corporate PC” to Connect to the F5 Big IP GUI
\sphinxurl{https://192.168.10.216}

\item {} 
Access\textgreater{}\textgreater{}Overview\textgreater{}\textgreater{}Active Sessions

\item {} 
Click on the session to view details
\begin{quote}

\sphinxincludegraphics[width=5.25486in,height=1.65269in]{{image22}.png}
\end{quote}

\item {} 
Log off using the windows start icon in the lower left corner

\end{enumerate}



\renewcommand{\indexname}{Index}

\backcoverpage

\end{document}